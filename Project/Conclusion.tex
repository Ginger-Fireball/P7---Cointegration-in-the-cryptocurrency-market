\chapter{Conclusion}
At the beginning of the project an introduction was given which included the desires for the project at hand were introduced and this conclusion will look to recap the findings in the project.\\
In Chapter \ref{chap:Coint} the concept of spurious regression and its issues were introduced before suggesting a potential solution, hence cointegration, which suggests the opportunity for creating a stationary process $\b{\beta}^\top\b{x}_t$. The project then concerns ways of testing whether time series share a cointegration relation where two different tests are introduced in Chapter \ref{chap:coint tests}. The first of which is the Engel Granger two-step approach, which in our case only consider the bi variate case. The first step is to reject the null hypothesis, which states that there is no cointegration, when this has been rejected step two is, creating an error correction model for each of the two variables. The second test is the Johansen test. Which test for the number of cointegrating relations, between multiple time series. The two tests has restrictions one of which is stationarity in the residuals obtained from a VAR model. The Augmented Dickey-Fuller test is used to check for stationarity where an introduction to Brownian motion was necessary since the residuals the ADF were applied to were estimated residuals. The calculations has not been computed since these values are readily accessible but sufficient theory for computing these is provided. \\\\
Throughout the application the goal was to use cointegration in a forecasting perspective and hence analyze whether cointegration tests are beneficial in forecasting setting. The Engle Granger resulted in four different combination, namely SOL \& ETH, XRP \& ETH, ETH \& SOL and XRP \& SOL. It was then chosen to proceed with the case in which the cointegration was found in both direction. A VECM was build using ETH and SOL with a lag order of six. While the Johansen indicated the presence of two cointegration relations. Then the Johansen VECM was transformed into a VAR(7). The models were then used to forecast and evaluated on both a 20-day-ahead prediction plot and mean errors for the first 5-day-ahead predictions. The 20-day-ahead prediction plots. In the case of the Engle Granger model, it could be seen that the prediction somewhat followed the actual values and captured some of the correct movement, especially in the start. While the prediction measured on all three MAE, RMSE and MAPE were mediocre at best. The 20-day-ahead prediction plot for Johansen showed even better movement capturing, particularly for bitcoin and ethereum, while less so for ripple and solana. In both models all actual prices where captured by the confidence interval. The Johansen model measured on mean errors, performed quite good. The errors was well below the Engel Granger, with one-day-ahead prediction for bitcoin and ethereum off by less than $1\%$. It was concluded that the Johansen model outperformed the Engle Granger model significantly, which were to be expected.

This section is rounded off by looking at the problem statement. It can be concluded that the econometric phenomena cointegration is present when analyzing cryptocurrencies and there are clear indications that this theory can be used in a forecasting framework and give satisfactory results. Finally the limitations for the Engle Granger model indicates that it is more beneficial to use a more complicated model such as the Johansen test provides.



