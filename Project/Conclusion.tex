\chapter{Conclusion}
At the beginning of the project an introduction was given and this conclusion will aim to recap the findings in the project.\\
In Chapter \ref{chap:Coint} the concept of spurious regression, and its issues, were introduced before suggesting a potential solution, hence cointegration, which suggests creating a stationary process $\b{\beta}^\top\b{x}_t$. The project then regards ways of testing whether time series share a cointegration relation, where two different tests are introduced in Chapter \ref{chap:coint tests}. The first of which is the Engle Granger two-step approach. The first step constructs a cointegration regression, and afterwards the error term is tested in order to confirm stationarity. When this is achieved, an VECM is constructed. The second test is the Johansen test, which tests for the number of cointegration relations, between multiple time series. The two tests have restrictions, one of which is stationary residuals obtained from a VAR model. The Augmented Dickey Fuller test is used to check for stationarity where an introduction to Brownian motion was necessary since the residuals were estimated residuals and hence compared to specific critical values. The calculations have not been computed since these values are readily accessible, but sufficient theory for computing these were provided. \\\\
Throughout the application the goal was to use cointegration in a forecasting perspective and hence analyze whether cointegration tests are beneficial in a forecasting setting. The Engle Granger resulted in four different combination, namely SOL \& ETH, XRP \& ETH, ETH \& SOL and XRP \& SOL. It were then chosen to proceed with the case where the cointegration was found in both directions. A VECM was constructed using SOL and ETH with a lag order of six. The Johansen test indicated the presence of two cointegration relations, and the VECM achieved was transformed into a VAR$(7)$. The models were then used to forecast, and evaluated on both 20-day-ahead prediction plots and accuracy measures for the first five-day-ahead predictions. In the case of the Engle Granger model, it could be seen that the 20-day-ahead predictions forecasting ability seemed lackluster, and the accuracy measures MAE, RMSE, and MAPE, were mediocre at best. The 20-day-ahead prediction plot for Johansen showed better movement capturing, particularly for BTC and ETH, while less so for XRP and SOL. In both models, all actual prices were captured by the confidence intervals. The Johansen model's accuracy measures were an improvement compared to the Engle Granger predictions and generally the errors are assumed quite acceptable in a forecasting matter, with a one-day-ahead prediction for BTC and ETH off by less than $1\%$. It was concluded that the Johansen model outperformed the Engle Granger model, which were to be expected.

\noindent This section is rounded off by looking at the problem statement.\\
\noindent It can be concluded that the econometric phenomena, cointegration, is present when analyzing cryptocurrencies and there are clear indications that this theory can be used in a forecasting framework and give satisfactory results. Finally, the limitations for the Engle Granger model indicates, that it is more beneficial to use a more complicated model such as the one the Johansen test provides.