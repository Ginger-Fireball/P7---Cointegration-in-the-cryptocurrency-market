\chapter{Discussion}
Looking back at the introduction and problem statement, this project has succeeded in describing, answering, and resolving some of the complications that could arise when forecasting the closing prices of cryptocurrencies. This discussion will aim to recap what has been achieved throughout the application, and in addition consider different possibilities of improving the model.

\section{Potential Model Enhancements}
Initially what should be considered is the different levels of values in cryptocurrencies where a solution such as logging the prices could eliminate some of the extremes. This process can although give skewed conclusions as to a log-transformed relation could occur even though the non-logged would not or vice versa because of the log being non-linear. Hence a cointegration or non cointegration relation between two logged prices might be obtained without the same relation seen for the prices in levels. It is common practice to, when looking at economic variables, taking the logarithm of the values but issues can arise in cointegration relations \cite{cointegrationloggedissues}.\\

\noindent In model validation, the normality of the residuals could not be confirmed which, for the purpose of the project, were assumed to be fulfilled anyways along with the heteroscedasticity assumption. This choice should of course be questioned and furthermore noted whether it affects the models forecasting abilities. The reason for maintaining this assumption is because the Johansen test assumes normality and since this were in fact not fulfilled, the Johansen test is not asymptotically reliable. There were although a silver lining in the sense that the Ljung-Box test did not indicate any autocorrelation in the residuals. This opens up the question whether a VAR model would have been favored. However, this is not the desire for the problem at hand and was therefore not considered. It could although have been an interesting approach to test whether forecasts using a VAR model would have been advantageous and resulted in more accurate predictions.\\

\noindent The data chosen are the four biggest non-stable coins based on market cap where three of the four (BTC, ETH, SOL) visually follows relatively the same pattern \ref{graph_in _levels}. It can be seen that the three currencies have an explosive growth followed by rapid decline resulting in very high peaks. Thus, there are significant shifts in the data. There are certainly indications that the cryptocurrencies analyzed are cointegrated. In addition, it could have been interesting to examine other time series which are, at least intuitively, cointegrated (such as the price of Mastercard stock compared with the price of Visa stock) to test whether the results can be compared to the findings in this project.\\

\noindent Another improvement could have been including other tests. There are multiple functions in R for just the Engle Granger test, in addition to the possibility of calculating it manually. Through brief testing, we have discovered that manual calculations differ from the results achieved using an R function. The reasoning behind these differences are unknown, and whether a better result would have been achieved through manual calculations is likewise unknown. Finally, there are also other cointegration tests and, since the Engle Granger test is at best mediocre at predicting it could have been interesting to check whether another test e.g. Phillips-Ouliaris would have been more accurate.\\

\noindent Another interesting approach could have been applying the Johansen test, which seems to be better than the Engle Granger test in a two variable relation.

\section{Model Validity}
The project has focused on two different cointegration tests, Engle Granger and Johansen, which has resulted in two different forecasting models. Engle Granger has its limitations, meaning it only considers a pairwise cointegration relation and thus only constructs a cointegration model between SOL and ETH. The model achieved through this process has been applied to a set of validation data and results in inadequate predictions. In Table \ref{table:SOL_ETH_MAE_RMSE_MAPE} the two-day-ahead predictions are the most accurate lying at 17.86\% and 29.22\% whilst the others lie between 19\%-20\% and 30\%-32\% overall for SOL and ETH respectively. Therefore, it can be argued that the forecasting model created by the Engle Granger test is mediocre at best.\\
The Johansen test has the advantage of being able to detect multiple cointegration relations and hence, if fitting, construct a VECM containing all the cryptocurrencies. The Johansen test suggests there are two cointegration relations, alluding that two forecasting models could be constructed. The initial improvement is the ability to forecast all the currencies examined. The big improvement is although seen when inspecting Table \ref{fig:johansen_RMSE_MAE_MAPE}, where the MAPE results are significantly better than for the Engle Granger model. For BTC the one-day-ahead prediction is only off by 0.71$\%$ which is assumed acceptable in a forecasting spectrum and for the five-day-ahead the MAPE is only 4.66$\%$.

\section{Reflection}
The project has examined whether four different cryptocurrencies are cointegrated. The concept of cointegration can be used in a variety of settings where certain stocks can be analyzed and clarified if they are cointegrated, as mentioned earlier. More macroeconomic and enviromental approaches could also be of interest, and other researchers have proven a statistical significant relation between GDP and CO$_2$ emissions \cite{A_Cointegration_Analysis_of_Real_GDP}, which in itself can be a very important observation.