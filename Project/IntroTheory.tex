\chapter{Introduction}
Spurious regression is a random relationship that can occur when analyzing different time series. It occurs when there seems to be a relation between two time series and when such relation appear, especially if the time series observed are non-stationary, there is a risk of it being a spurious regression. The issue with spurious regression is that it insinuates that two time series, which are not related, have a correlation.\\\\
\noindent Naturally, there are multiple cases where cointegration is present and a significant factor; hence cases where the relation is reliable, and one series has an impact on the other and vice versa. Cointegration often occurs when analyzing time series which are describing some economic movement such as assets. When analyzing time series, if cointegration is present, it means that two or more series have a correlation in the long-run. In general, this means that even though the different series does not exhibit a short-run correlation, then in the long-run, it is possible to find a combination which will not diverge from a common equilibrium. When series share a long-run equilibrium, they are cointegrated\cite{Intro_cointegration}.\\\\
There are several ways to determine whether series are cointegrated, and as mentioned, economic related time series are often cointegrated. Series such as product supply, demand will over time share an equilibrium.\\
This project will examine cointegration between price levels over time of different cryptocurrencies. Cryptocurrencies are digitally exchanged assets, and one of their perks is the lack of need for a trusted third party, such as a government or financial institute. In this project, we will examine whether some or all of the four chosen cryptocurrencies are cointegrated.\\
Cryptocurrencies cointegration relation, which could be seen by plotting the prices in levels, could somewhat be explained by general market consensus regarding both the interest and varying private equity over time \cite{Coinmarket}. 
Bitcoin is by far the most well-known cryptocurrency, and since its publication in 2008 it has risen to a value of over 100.000\$ with a year-to-date percentage change of over 135\%. Although Bitcoin is the most expensive and well-known cryptocurrency, the existence of other cryptocurrencies is important to create a decentralized financial system and avoid a monopoly.\\
When having different cryptocurrencies, a natural question is whether these have a long-run relationship. If they do, this could be useful in various regards, such as the ability to forecast prices of different coins. Over the years investors have been attracted into cryptocurrency, investing because of many lucrative factors such as a quickly expanding market with new coins emerging all the time, a way of investing in currencies which are not printed or stored in a bank, and maybe most importantly, the volatility of these prices. Therefore, being able to predict the value of different cryptocurrencies in the future is a desired feature \cite{investopedia.} \cite{Coinmarket}.


\newpage
This desire leads to the aim of this project, which is stated below:\\\\
\noindent\makebox[\linewidth]{\rule{\textwidth}{0.4pt}}
\underline{Problem statement:}\\\\
\textbf{\textit{1. What is cointegration and is it present when analyzing cryptocurrencies?}}\\\\
\textbf{\textit{2. Is cointegration useful when forecasting cryptocurrencies?}}\\\\
\textbf{\textit{3. What different results are obtained when using the Engle Granger or the Johansen cointegration test?}}\\
\noindent\makebox[\linewidth]{\rule{\textwidth}{0.4pt}}
Throughout this project, time series theory is used when analyzing cointegration between four different cryptocurrencies. A section containing the necessary theory sets the groundwork for answering the three questions above, providing a formal definition of cointegration along with other definitions and theorems.\\
We apply the theory on the closing prices of the four cryptocurrencies Bitcoin, Etherium, Solana and Ripple over the time span 2020-12-10 to 2024-10-10, where 10\% of this data has been used for validation. After analyzing if the currencies are cointegrated, we will use econometric theory to try and predict the movements of the examined cryptocurrencies.\\\\
The project will end with a conclusion where the problem statement will be answered.



