\chapter{Introduction}
Spurious regression is a random relationship which can occur when analyzing different time series. It occurs when there seems to be a relation between two time series and when such relation occurs, especially if the time series observed are non-stationary, there is a risk for it being a spurious regression. The issue with spurious regression is that it insinuates that two time series, which are not related, has a correlation.\\\\
\noindent Naturally there are multiple cases, where cointegration is present and a significant factor, hence cases where the relation is reliable and one series has an impact on the other and vice versa. Cointegration is often occurring when analyzing time series which are describing some economic movement such as assets. When analyzing time series then, if cointegration is present, it means that two or more series has a correlation in the long run. In general this means that even though the different series does not exhibit a short run correlation then in the long run they will not diverge from a common equilibrium. When series share a long run equilibrium they are cointegrated\cite{Intro_cointegration}.\cite{Intro_cointegration}\\\\
There are several ways to determine wether series are cointegrated and as mentioned economic related time series are often cointegrated. Series such as prduct supply, demand and time will over time share an equilibrium.\\
This project will examine cointegration between price levels over time of different cryptocurrencies. Cryptocurrencies are digitally exchanged assets and one of their perks is the lack of need for a trusted third party such as a government or financial institute. In this project, we will examine whether some or all of the four chosen cryptocurrencies are cointegrated.\\
By just plotting the prices over time, three of the four examined time exhibit a somewhat identical graph which could be caused by a general market consensus regarding both the interest, and varying private equity over time \cite{Coinmarket}. Bitcoin is by far the most well-known crytovaluta and has, since its publication in 2008 risen to a value of over 100.000\$ and the year-to-date percentage change is over 130\%. Although Bitcoin is the most expensive and well-known cryptovaluta, the existence of other cryptovalutas are important to create a decentralized financial system and avoid a monopole.\\
Having different cryptovalutas then a natural wondering is whether these have a long run realationship. If they have this could be useful in different regards such as the ability to forecast prices of different valutas. Over the years investors has been lured into cryptovaluta investing because of many lucrative factors such as a quickly expanding market with new valutas coming up all the time, a way of investing in valutas which are not printed or stored in a bank and maybe most importantly, the volatility of these prices. Therefore being able to predict the value of different cryptovalutas in the future is a desired feature \cite{investopedia.}.

\newpage

This desire leads to the aim of this project which is stated below:\\\\
\noindent\makebox[\linewidth]{\rule{\textwidth}{0.4pt}}
\underline{Problem Statement:}\\\\
\textbf{\textit{1. What is cointegration and is it present when analyzing crypto currencies?}}\\\\
\textbf{\textit{2. How is cointegration between crypto currencies used when forecasting these as time series?}}\\\\
\textbf{\textit{3. What different results are obtained when using Engel Granger or Johansen cointegration test?}}\\
\noindent\makebox[\linewidth]{\rule{\textwidth}{0.4pt}}
Throughout this project time series theory is used when analyzing cointegration between four different crypto currencies. A section containing the needed theory sets the groundwork for answering the three questions above where a formal definition of cointegration among other definitions and theorems are given.\\
We apply the theory on the four crypto currencies bitcoin, etherium, solana, ripple on the closing prices over the span 2020-12-10 until 2024-10-10 where 10\% of this has been used for validation data. After analyzing wether the currencies are cointegrated we will use forecasting theory in order to try and predict the movements of the examined crypto currencies.\\\\\\\\
The project will end using a conclusion where the problem statement will be answered.



