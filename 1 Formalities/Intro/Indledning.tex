\noindent
\chapter*{Introduction}
\label{Introduction}

In time series analysis, where multiple non-stationary series are analysed, there can sometimes be observed a relationship between them. However, sometimes the observed relationship is a phenomenon known as spurious regression. This is a case where a random relationship is observed between non-related time series. The test statistics in this case suggest a relationship where there is none. When the case is that a relationship is expected between the time series, then cointegration will be present.\\
Cointegration is mainly used when analysing economic time series and the relationship between the different assets is of interest. The focus is to check for a long run relationship, meaning that the time series has a long run equilibrium, which they will only diverge from in the short term.\\

\noindent Cointegration is often used with economic time series because a relationship is expected. A lot of cases make intuitively sense, take e.g. housing prices and rent prices. The expectation is that when one goes up the other does as well. This project will investigate the relationship between different cryptocurrencies. The biggest one by far is Bitcoin measured by market capitalization and today boasts a market capitalization of over a trillion dollars, it was first released as open-source back in 2009 and since its inception, the cryptocurrency space has had explosive growth and adaptation, both in number of coins, market capitalization and transaction volume. The second largest is Ethereum. These two have been chosen together with Solana, XRP and Cardano. This leads to the following problem statement.

\begin{note}Problem statement
    
\end{note}