\documentclass[11pt,a4paper,oneside,openany,english]{book}
%har ændre twoside -> oneside
% Openany, gør at den kan starte kapitel på hvilken somhelst side, openright så kan kapitel kun starte på højre side ( og der vil komme blanke sider mellem kapitler.


\usepackage{subcaption} % Required for subfigure

\usepackage[integrals]{wasysym}
\usepackage[makeroom]{cancel}
\usepackage{relsize}
\usepackage{stmaryrd}
\usepackage{tocloft}

\usepackage[export]{adjustbox}
\usepackage[utf8]{inputenc}					% Input-indkodning af tegnsaet (UTF8)
\usepackage[english,danish]{babel}
				% Dokumentets sprog
\usepackage[T1]{fontenc}					% Output-indkodning af tegnsaet (T1)
\usepackage{ragged2e,anyfontsize}

\usepackage{multirow}                		% Fletning af raekker og kolonner (\multicolumn og \multirow)

\usepackage{mathtools}						% Andre matematik- og tegnudvidelser¨
\usepackage{bigints}
\usepackage{mathrsfs}

\usepackage[square,sort,comma,numbers]{natbib}				
\bibliographystyle{apalike}			% Udseende af litteraturlisten.

% \setcitestyle{square}

\usepackage{pgfplots}
\pgfplotsset{compat=1.15}

\usepackage[nottoc,numbib]{tocbibind}
% bibliografi i indfoldsfortegnelse

% ¤¤ Opsaetning af figur- og tabeltekst ¤¤ %
\usepackage[font=footnotesize,labelfont={bf},format=hang]{caption}

\usepackage{flafter}						% Soerger for at floats ikke optraeder i teksten foer deres reference
\let\newfloat\relax 						% Justering mellem float-pakken og memoir
\usepackage{float}
% Muliggoer eksakt placering af floats, f.eks. \begin{figure}[H]


\usepackage{changepage}


%%%% ORDDELING %%%%

\hyphenation{In-te-res-se e-le-ment}
\hyphenation{dis-con-nec-ted}

%   Sidehoved/sidefod
\usepackage{lastpage}
\usepackage{fancyhdr}
\setlength{\headheight}{15pt}
\pagestyle{fancy}
\fancyhf{}
\renewcommand{\chaptermark}[1]{ \markboth{\thechapter.\ #1}{}}
\fancyheadoffset{0pt}
\rhead{\nouppercase \leftmark}
%\chead{Gruppe 4.219}
\lhead{Aalborg Universitet}
\lfoot{ }
\cfoot{\thepage}
\usepackage{etoolbox}
\patchcmd{\chapter}{plain}{fancy}{}{}

\usepackage{boldline}

%   Kapiteludseende
\usepackage{xcolor}
\usepackage{titlesec, blindtext, color}
\definecolor{gray75}{gray}{0.75}
\newcommand{\hsp}{\hspace{20pt}}
\titleformat{\chapter}[hang]{\Huge\bfseries}{\thechapter\hsp\textcolor{gray75}{|}\hsp}{0pt}{\LARGE\bfseries}
\titlespacing*{\chapter}{10pt}{0pt}{10pt}

%Bibliografi og TOC
\addto\captionsenglish{
	\renewcommand\contentsname{Table of contents}	
	\renewcommand{\bibname}{Bibliography}
}
\setcounter{tocdepth}{3}
\setcounter{secnumdepth}{3}

%---------------------------------
% Pretty much all of the ams maths packages
\usepackage{amsmath,amsthm,amssymb,amsfonts}

% Allows you to manipulate the page a bit
\usepackage[a4paper]{geometry}

% Pulls the page out a bit - makes it look better (in my opinion)
\usepackage{a4wide}

% Removes paragraph indentation (not needed most of the time now)


% Allows inclusion of graphics easily and configurably
\usepackage{graphicx}
% Provides ways to make nice looking tables
\usepackage{booktabs}


% Allows you to rotate tables and figures
\usepackage{rotating}

% Allows shading of table cells
\usepackage{colortbl}

% Define a simple command to use at the start of a table row to make it have a shaded background
\newcommand{\gray}{\rowcolor[gray]{.9}}

\usepackage{textcomp}

% Typesets URLs sensibly - with tt font, clickable in PDFs, and not breaking across lines
\usepackage{url}

% Makes references hyperlinks in PDF output
\usepackage[hidelinks]{hyperref}

% Makes it possible to ref an unnumberede section
\usepackage{hyperref}
\usepackage{nameref}

% Lets you use different kinds of matrixes
\usepackage{nicematrix}

%---------- ekstra telføjelser ------
\usepackage{mdframed}


%TikZ
\usepackage{tikz}
%\usetikzlibrary{arrows, petri, topaths,graphs,graphs.standard}
%\usetikzlibrary{arrows.meta}
%\usepackage{tkz-berge}
\usepackage{subfig}
%\usepackage{verbatim}
%\usetikzlibrary{calc}
\usepackage{pgfplots}
\usepackage{pgfplotstable}
\pgfplotsset{compat = newest}

%---- pseudocode 
\usepackage{algorithm}
\usepackage[noend]{algpseudocode}


\renewcommand{\gets}{:=}

\renewcommand{\a}{\alpha}
\newcommand{\y}{\gamma}
%\newcommand{*}{\cdot}
\def\*{\cdot}
\newcommand{\BO}{\mathcal{O}}



%Citationstegn
\usepackage[utf8]{inputenc}
\usepackage{dirtytalk}


\usepackage{marvosym}



%Alt til definitioner, sætninger osv.
\usepackage{framed}
\definecolor{myGray}{HTML}{F9F9F9}

%--------

\renewenvironment{leftbar}[4][\hsize]
{
    \def\FrameCommand
    {
        {\color{#2}\vrule width #4pt}
        \hspace{-8pt}
        \fboxsep=\FrameSep\colorbox{#3}
    }
    \MakeFramed{\hsize#1\advance\hsize-\width\FrameRestore}
}
{\endMakeFramed}

\usepackage{array}
\usepackage{makecell}

\renewcommand\theadalign{bc}
\renewcommand\theadfont{\bfseries}
\renewcommand\theadgape{\Gape[4pt]}
\renewcommand\cellgape{\Gape[4pt]}

\newcommand{\vc}[1]{\textbf{\textit{#1}}}
%----------


\newtheoremstyle{beviss}% name of the style to be used
  {10pt}% measure of space to leave above the theorem. E.g.: 3pt
  {\topsep}% measure of space to leave below the theorem. E.g.: 3pt
  {}% name of font to use in the body of the theorem
  {0pt}% measure of space to indent
  {\itshape}% name of head font
  {:\\}% punctuation between head and body
  { }% space after theorem head; " " = normal interword space
  {\thmname{#1}}

\theoremstyle{definition}
\newtheorem{pro}{Property}[chapter]
\newtheorem{setn}{Theorem}[chapter]%Sætning
\newtheorem{lem}{Lemma}[chapter]
\newtheorem{propo}{Proposition}[chapter]
\newtheorem{koro}{Corollary}[chapter]
\newtheorem{defn}{Definition}[chapter]
\newtheorem{culu}{Condition}[chapter] 
\newtheorem{exmp}{Example}[chapter]
\newtheorem{met}{Metode}[chapter]
\newtheorem*{bema}{Remark}

\theoremstyle{beviss}
\newtheorem{bevis}{\textbf{Proof}}
\AtEndEnvironment{bevis}{\null\hfill$\blacksquare$}%

\theoremstyle{beviss}
\newtheorem{modstrid}{\textbf{Proof}}
\AtEndEnvironment{modstrid}{\null\hfill\Lightning}%

%----
\newenvironment{property}[1]
    {\begin{leftbar}{black}{myGray}{3}
    \begin{pro}#1 \\
        }{
    \end{pro}
    \end{leftbar}
    }

\newenvironment{note}
    {\begin{leftbar}{black}{myGray}{3}
        }{
    \end{leftbar}
    }

\newenvironment{condition}[1]
    {\begin{leftbar}{black}{myGray}{3}
    \begin{culu}#1\\
        }{
    \end{culu}
    \end{leftbar}
    }



\newenvironment{thm}[1]
    {\begin{leftbar}{black}{myGray}{3}
    \begin{setn}#1\\
        }{
    \end{setn}
    \end{leftbar}
    }
    
\newenvironment{lemma}[1]
    {\begin{leftbar}{black}{myGray}{3}
    \begin{lem}#1\\
        }{
    \end{lem}
    \end{leftbar}
    }
  
\newenvironment{defi}[1]
    {\begin{leftbar}{black}{myGray}{3}
    \begin{defn}#1\\
        }{
    \end{defn}
    \end{leftbar}
    }
    
    \newenvironment{cor}[1]
    {\begin{leftbar}{black}{myGray}{3}
    \begin{koro}#1\\
        }{
    \end{koro}
    \end{leftbar}
    }
    
\newenvironment{ekse}[1]
    {\begin{leftbar}{gray}{white}{2}
    \begin{exmp}#1\\
        }{
    \end{exmp}
    \end{leftbar}
    }

\newenvironment{prop}[1]
    {\begin{leftbar}{black}{myGray}{3}
    \begin{propo}#1 \\
        }{
    \end{propo}
    \end{leftbar}
    }
    
\newenvironment{meto}[1]
    {\begin{leftbar}{gray}{myGray}{3}
    \begin{met}#1\\
        }{
    \end{met}
    \end{leftbar}
    }
    
\usepackage{enumitem} 

\usepackage{caption}
\usepackage{subcaption}
\renewcommand\qedsymbol{$\blacksquare$}

%Floats
\usepackage{placeins}

%Matricer
\usepackage{blkarray}
\newcommand{\matindex}[1]{\mbox{\scriptsize#1}}
\newenvironment{amatrix}[1]{%
  \left[\begin{array}{@{}*{#1}{c}|c@{}}
}{%
  \end{array}\right]
}
\newenvironment{cmatrix}[1]{%
  \left[\begin{array}{@{}*{#1}{c}| @{}*{#1}{c}}
}{%
  \end{array}\right]
}

\newcommand{\ts}{\textsuperscript}


\makeatletter
\renewcommand\tableofcontents{%
  \section*{\contentsname}% or \chapter* but that looks ugly
  \pagestyle{plain}%
  \@starttoc{toc}}
\makeatother


%shortcuts
\newcommand{\trk}{\nabla} %Laver trekant til gradientvektor
\newcommand{\m}{\cdot} %Hurtigt gangetegn
\newcommand{\df}{\frac{\partial f}{\partial x}} %Differentiation df/dx
\newcommand{\dx}{\frac{d x}{d t}}
\newcommand{\disneyd}{\mathcal{D}}
\newcommand{\Lag}{\mathcal{L}}
\newcommand{\R}{\mathbb{R}}
\newcommand{\F}{\mathbb{F}}
\newcommand{\C}{\mathbb{C}}
\newcommand{\Z}{\mathbb{Z}}
\newcommand{\N}{\mathbb{N}}
\newcommand{\ML}{\mathbb{L}}
\newcommand{\K}{\mathbb{K}}
\newcommand{\Q}{\mathbb{Q}}
\newcommand{\tr}{\text{tr}}
\newcommand\norm[1]{\left\lVert#1\right\rVert} % Norm
\newcommand\fnorm[1]{\left\lVert#1\right\rVert_F} % Frobeniusnorm
\newcommand{\NA}{\mathcal{N}_A} % Løsningsrum 
\newcommand{\epsi}{\varepsilon}
\newcommand{\E}{\mathcal{E}}
\newcommand{\M}{\mathcal{M}}
\newcommand\ip[1]{\left\langle#1\right\rangle} 
\newcommand{\x}{\bm{x}}
\renewcommand{\y}{\bm{y}}
\renewcommand{\v}{\bm{v}}
\newcommand{\w}{\bm{w}}
\renewcommand{\u}{\bm{u}}
\newcommand{\T}{\mathcal{T}}
\newcommand{\A}{\mathcal{A}}
\newcommand{\0}{\bm{0}}
\renewcommand{\t}{\bm{t}}
\newcommand{\Sym}{\text{Sym}}
\newcommand{\Isom}{\text{Isom}}
\newcommand{\om}{\bm{\omega }}


%ting der gør det nemmere JP
\newcommand{\bbeta}{\boldsymbol{\beta}} % gør beta tyk
\newcommand{\by}{\textbf{y}}
\newcommand{\bX}{\textbf{X}}
\newcommand{\bx}{\textbf{x}}
\newcommand{\bi}{\textbf{i}}
\newcommand{\btheta}{\boldsymbol{\theta}}
\newcommand{\bmu}{\boldsymbol{\mu}} %gør mu tyk
\newcommand{\bY}{\textbf{Y}} %gør y fed
\newcommand{\pause}{\hspace*{1pt}\hfill}




%Vectors
\newcommand{\vecto}[2]{\begin{bmatrix}#1\\#2\end{bmatrix}}
\newcommand{\vectre}[3]{\begin{bmatrix}#1\\#2\\#3\end{bmatrix}}

%Prikprodukt
\makeatletter
\newcommand*\prik{\mathpalette\prik@{.5}}
\newcommand*\prik@[2]{\mathbin{\vcenter{\hbox{\D{#2}{$\m@th#1\bullet$}}}}}
\makeatother


%andre packages
\usepackage{tasks}
\usepackage{tabularx}
\usepackage{pythonhighlight}
\lstset{
  numbers=left,
  stepnumber=1,    
  firstnumber=last,
  numberfirstline=true
}
\usepackage{listings}

%Align med bogstaver
\newenvironment{alphalign}
 {%
  % save the value of equation
  \setcounter{parentequation}{\value{equation}}%
  % reset equation and make it \alph
  \setcounter{equation}{0}%
  \renewcommand{\theequation}{\alph{equation}}%
  % start a standard align
  \align}
 {%
  % end align
  \endalign
  % restore equation
  \setcounter{equation}{\value{parentequation}}}
 
 
\makeatletter
\newcommand{\vx}{\vec{x}\@ifnextchar{^}{\,}{}
\makeatother

\makeatletter
\newcommand{\vl} \vec{\lambda}\@ifnextchar{^}{\,}{}}
\makeatother
\usepackage{esvect}


\newcommand{\Mod}[1]{\ (\mathrm{mod}\ #1)}

\usepackage{bm}

% %Appendix
%
\usepackage[title,titletoc,toc]{appendix}
%Change name in TOC
%\renewcommand\appendixtocname{Appendikser}

%Change Appendix in the language package to be danish
% \addto\captionsdanish{%
%   \renewcommand\appendixname{Appendikser}
%   \renewcommand\appendixpagename{Appendikser}
% } 

%Restriction of operator

\newcommand\res[2]{{% we make the whole thing an ordinary symbol
  \left.\kern-\nulldelimiterspace % automatically resize the bar with \right
  #1 % the function
  \vphantom{\big|} % pretend it's a little taller at normal size
  \right|_{#2} % this is the delimiter
  }}
  
  \usepackage{bm}
  \usepackage{bbm}
  
  \usetikzlibrary{calc, automata, chains, arrows.meta, quotes}
  
\usepackage{blkarray, bigstrut}
\usepackage{icomma}
\usepackage{booktabs}
\usepackage{IEEEtrantools}

%Commands to make life easier
\newcommand{\tw}{\textcolor{white}}
\newcommand{\phm}{\phantom{-}}

\usepackage{multirow}


\newcommand\Tophat[1]{\hat{#1}}
\newcommand{\bigO}{\mathcal{O}}
\newcommand{\ud}{\,\mathrm{d}x}
\DeclareMathSymbol{*}{\mathbin}{symbols}{"01}
%Den her lader os kommenterer
\usepackage{comment}
\usepackage{emptypage}
\usepackage{tikzit}
\input{sample.tikzstyles}
\usetikzlibrary{positioning, arrows.meta}
\usepgfplotslibrary{fillbetween}

\renewcommand{\a}{\alpha}

\definecolor{codegray}{rgb}{0.5,0.5,0.5}
\definecolor{AAUblue1}{RGB}{ 33, 26, 82}
\definecolor{AAUblue2}{RGB}{ 89, 79,191}
\definecolor{AAUgrey1}{RGB}{ 84, 97,110}
\definecolor{AAUgrey2}{RGB}{104,119,132}
\definecolor{AAUgrey3}{RGB}{162,172,182}
\definecolor{AAUgrey4}{RGB}{222,223,226}
\definecolor{AAUgreen}{RGB}{157,187, 29}
\definecolor{AAUteal} {RGB}{ 94,150,149}
\definecolor{AAUred}  {RGB}{223,103, 82}
\definecolor{codegreen}{rgb}{0,0.6,0}
\definecolor{codegray}{rgb}{0.5,0.5,0.5}
\definecolor{codepurple}{rgb}{0.8,0,0.5}
\definecolor{codegray}{gray}{0.4}


\lstdefinestyle{custompy}{
  belowcaptionskip=\baselineskip,
  breaklines=true,
  language=Python,
  showstringspaces=false,
  basicstyle=\footnotesize\ttfamily,
  keywordstyle=\bfseries\color{black},
  numberstyle=\tiny\color{black},
  commentstyle=\itshape\color{black},
  identifierstyle=\color{black},
  frame=lines,
  stringstyle=\color{black},
    breakatwhitespace=false,         
    breaklines=true,                 
    captionpos=b,                    
    keepspaces=true,                 
    numbers=none,                    
    numbersep=15pt,                  
    showspaces=false,                
    showtabs=false,                  
    tabsize=2
}

\lstdefinestyle{jpstyle}{
  keywordstyle=\bfseries\color{codegreen},
    stringstyle=\color{codegreen},
    commentstyle=\color{codepurple},
    numberstyle=\tiny\color{codegray},
    basicstyle=\ttfamily\footnotesize,
    breakatwhitespace=false,         
    breaklines=true,                 
    captionpos=b,                    
    keepspaces=true,                 
    numbers=left,                    
    numbersep=5pt,                  
    showspaces=false,                
    showstringspaces=false,
    showtabs=false,                  
    tabsize=2
}


\renewcommand{\lstlistingname}{Kildekode}
\renewcommand{\lstlistlistingname}{Kildekode}

\hyphenation{før-ste-or-dens}
\hyphenation{i-ma-gi-næ-re}
\hyphenation{kom-bi-na-ti-o-ner-ne}
\hyphenation{kon-sis-tent}
\hyphenation{ud-vik-ling}
\hyphenation{co-ro-na-e-pi-de-mi-en}
\hyphenation{ø-ko-no-mi}
\hyphenation{mul-ti-pli-ka-tor-ac-ce-le-ra-tor-mo-del}
%\hyphenation{sam-funds-ø-ko-no-mis-ke}'


%Basse's tegn
\newcommand{\indep}{\perp\!\!\perp}



\renewcommand{\b}[1]{\boldsymbol{#1}}